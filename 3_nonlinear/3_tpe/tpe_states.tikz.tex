%\documentclass[margin=0mm]{standalone}
%\input{../../tikz_header}
%
%\usepackage{braket}
%
%\begin{document}
%


\begin{tikzpicture}
%\useasboundingbox (0.,0) rectangle (10.6,8.3);

\tikzmath{\x = 0.; \y0 = 0.3; \y1 = 1.2; \y2 = 1.8; \dy = 0.25; \len = 1;};

%\draw (0.,0) rectangle (5.1,5.1);


\draw (\x,\y0) -- (\x + \len,\y0) node [right] {$\ket{g}$ initial state $\ket{i} = \ket{g, n}$};

\draw  (\x,\y1) -- (\x + \len,\y1)  node  [right] {$\ket{e'}$ intermediate state $\ket{x} = \ket{e', n-1}$};

\draw [dashed] (\x,\y2 ) -- (\x + \len,\y2)  ;

\draw  (\x,\y2 + \dy) -- (\x + \len,\y2 + \dy)  node  [right] {$\ket{e}$ final state $\ket{f} =\ket{e, n-2}$};

\tikzmath{\midml=(\y2 - \y0)/2;};

\draw[->] (\x + \len /2, \y0) --  (\x + \len /2, \y0 + \midml);
\draw[->] (\x + \len /2, \y0 + \midml) --  (\x + \len /2, \y0 +2 *  \midml);

\end{tikzpicture}

%\end{document}