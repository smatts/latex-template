%\documentclass[margin=0mm]{standalone}
%% everything that is connected to tikz


\newcommand{\inputtikz}[1]{%

 \tikzexternalenable
  \tikzsetnextfilename{#1}%
  \input{#1.tikz}%
  \tikzexternaldisable

}

\usepackage{tikz,tikz-3dplot}
\usepackage{standalone}
\usepackage{pgfplots,pgfplotstable}
\usepackage{tikzorbital}
 \usepackage{tikzsymbols}
 \usepackage{currfile,hyperxmp}
\usepackage{blochsphere}


\usetikzlibrary{math,matrix,fit,positioning,intersections}
\usetikzlibrary{patterns,decorations.pathmorphing,decorations.markings}
\usetikzlibrary{calc}
\usetikzlibrary{arrows.meta} %needed tikz library
\usetikzlibrary{quotes,angles}

 \pgfplotsset{compat=newest}
\usepgfplotslibrary{groupplots}

\tikzset{>=latex}


\pgfplotsset{
tufte line/.style={
    axis line style={draw opacity=0},
    ytick=\empty,
    axis x line*=bottom,
    x axis line style={
      draw opacity=1,
      gray,
      thick
},
 %   yticklabel=\pgfmathprintnumber{\tick}
  }
  }
%
%%\tikzset{
%%mymat/.style={
%%    matrix of math nodes,
%%    left delimiter=|, right delimiter=|,
%%    align=center,
%%    column sep=-\pgflinewidth,
%%}
%%,mymats/.style={
%%    mymat,
%%    nodes={draw,fill=#1}
%%} 
% }
% 
%\newcommand{\myarrow}[5]{\draw[#4](#1.south -| #2)  -- ++(#3 :6mm) node[above,pos=0.55]{$#5$};
%} 
%
%\newcommand{\interactLp}[3]{\myarrow{#1-#2-1}{#1.west}{-135}{<-}{#3}} 
%\newcommand{\interactLm}[3]{\myarrow{#1-#2-1}{#1.west}{+135}{->}{#3}} 
%\newcommand{\interactRp}[3]{\myarrow{#1-#2-2}{#1.east}{ -45}{<-}{#3}} 
%\newcommand{\interactRm}[3]{\myarrow{#1-#2-2}{#1.east}{ +45}{->}{#3}}  
%
%\newcommand{\interactout}[2]{\myarrow{#1-1-1}{#1.west}{+135}{->,dashed}{#2}} 
%
%
%

%
%\usepackage{braket}
%
%\begin{document}
%


\begin{tikzpicture}
%\useasboundingbox (0.,0) rectangle (10.6,8.3);

\tikzmath{\x = 0.; \y0 = 0.3; \y1 = 1.2; \y2 = 1.8; \dy = 0.25; \len = 1;};

%\draw (0.,0) rectangle (5.1,5.1);


\draw (\x,\y0) -- (\x + \len,\y0) node [right] {$\ket{g}$ initial state $\ket{i} = \ket{g, n}$};

\draw  (\x,\y1) -- (\x + \len,\y1)  node  [right] {$\ket{e'}$ intermediate state $\ket{x} = \ket{e', n-1}$};

\draw [dashed] (\x,\y2 ) -- (\x + \len,\y2)  ;

\draw  (\x,\y2 + \dy) -- (\x + \len,\y2 + \dy)  node  [right] {$\ket{e}$ final state $\ket{f} =\ket{e, n-2}$};

\tikzmath{\midml=(\y2 - \y0)/2;};

\draw[->] (\x + \len /2, \y0) --  (\x + \len /2, \y0 + \midml);
\draw[->] (\x + \len /2, \y0 + \midml) --  (\x + \len /2, \y0 +2 *  \midml);

\end{tikzpicture}

%\end{document}